\documentclass[english]{article}
\usepackage[utf8]{inputenc}
\usepackage{babel}
\usepackage{amsmath}

\begin{document}
\section{Angle calculation}
We calculate the angle by assuming that the ellipse (ear) we found was a circle
in the xy-plane that was rotated about the y axis and then projected into the
xy-plane again. The projection should theoretically be a proper camera
projection but for simplification purposes we assume a parallel projection.

We can discard the translation and rotation of our ellipse. Our ellipse is the
projection of a circle that was rotated about the y axis into the XY-plane.

We should be able to calculate the rotation angle from just four points at the
ends of the major and minor axes respectively.

A rotation about the y axis only changes the x coordinate (and z, but we cannot
determine the z coordinates from the image):

\begin{align*}
  x &= \cos{\theta} \cdot a \sin{\alpha} \\
  x_{1} = b &= \cos{\theta} \cdot a \sin{\frac{\pi}{2}} \\
  b &= \cos{\theta} \cdot a \\
  x_{2} = -b &= \cos{\theta} \cdot a \sin{\frac{3 \pi}{2}} = \cos{\theta} \cdot (-a) \\
  \theta &= \arccos{\frac{b}{a}}
\end{align*}

\end{document}
